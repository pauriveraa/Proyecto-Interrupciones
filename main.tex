\documentclass{article}
\usepackage[utf8]{inputenc}

\title{Proyecto de investigación Interrupciones}
\author{paubla.rivera }
\date{July 2020}

\begin{document}

\maketitle

\section{RESUMEN}
In the following work we talk about interruptions at the microprocessor level, the way in which devices communicate with the processor through them to have assigned a single line capable of alerting the CPU when it is required to perform an operation, the types that exist. and its applications.


\section{OBJETIVOS}
2.1 Identificar la utilidad del manejo de interrupciones. \\
2.2 Reconocer los distintos tipos de interrupciones. \\
2.3 Aplicar interrupciones a través de la plataforma arduino.


\section{INTRODUCCIÓN}
Dentro de la ciencia de la tecnología se estudia el tratamiento de la información y los sistemas que se encargan de su procesamiento. Uno de los mayores avances de la tecnología son las computadoras y  detrás de estos hay un mundo por explorar, y  en esta ocasión se hablará del uso de interrupciones que surgen de las necesidades que tienen los dispositivos periféricos de enviar información al procesador principal de un sistema de comunicación. Todos los dispositivos que deseen comunicarse con el procesador por medio de interrupciones deben tener asignada una línea única capaz de avisar al CPU cuando le requiere para realizar una operación. Esta línea se denomina IRQ y llegan al controlador de interrupciones que puede estar integrado en el procesador principal o ser un circuito separado conectado al mismo.[1]



\section{	MICROPROCESADORES Y MICROCONTROLADORES}

Los microprocesadores se han desarrollado fundamentalmente orientados al mercado de los ordenadores personales y las estaciones de trabajo, pues allí se requiere una elevada potencia de cálculo, el manejo de gran cantidad de memoria y una gran velocidad de procesamiento. Mientras que los microcontroladores están concebidos fundamentalmente para ser utilizados en aplicaciones puntuales, es decir, aplicaciones donde el microcontrolador debe realizar un pequeño número de tareas, al menor costo posible.[2]

\section{DEFINICIÓN DE INTERRUPCIÓN}

Una interrupción es un mecanismo que permite ejecutar un bloque de instrucciones interrumpiendo la ejecución de un programa, y luego restablecer la ejecución del mismo sin afectarlo directamente. De este modo un programa puede ser interrumpido temporalmente para atender alguna necesidad urgente del computador y luego continuar su ejecución como si nada hubiera pasado.
Generalmente se aplica para realizar tareas elementales asincrónicas en el computador tales como responder al teclado, escribir en la pantalla, leer y escribir archivos. Podemos considerar una tarea asincrónica como aquella que es solicitada sin previo aviso y aleatoriamente desde el punto de vista del computador.  [3]

\section{INICIOS}

La primera técnica que se empleó para enviar información al procesador  fue el polling, que consistía en que el propio procesador se encargara de sondear los dispositivos periféricos cada cierto tiempo para averiguar si tenía pendiente alguna comunicación para él. Este método presentaba el inconveniente de ser muy ineficiente, ya que el procesador consumía constantemente tiempo y recursos en realizar estas instrucciones de sondeo. [1]
El mecanismo de interrupciones fue la solución que permitió al procesador desentenderse de esta problemática, y delegar en el dispositivo periférico la responsabilidad de comunicarse con él cuando lo necesitara.



\section{TIPOS DE INTERRUPCIONES}
Existen distintas formas de generar interrupciones a tareas que realice un dispositivo, estas son: \\
7.1 Interrupciones de hardware: Estas son asíncronas a la ejecución del procesador, es decir, se pueden producir en cualquier momento independientemente de lo que esté haciendo el CPU en ese momento. Las causas que las producen son externas al procesador y a menudo suelen estar ligadas con los distintos dispositivos de entrada o salida. En la mayoría de las CPUs la respuesta a una interrupción consta de los siguientes pasos:
El dispositivo de hardware genera el pulso o señal de petición de interrupción, el controlador de Interrupciones Programables prioriza la petición de interrupción en relación con las demás peticiones que podrían haberse emitido de forma simultánea (o estar pendientes) y emite la petición de interrupción al procesador; si las interrupciones están habilitadas, la CPU responde con un ciclo de bus de reconocimiento de interrupción; en respuesta al reconocimiento de la CPU, el dispositivo externo (o el PIC si estuviese presente) sitúa un vector de interrupción en el bus de datos; la CPU lee el vector y lo utiliza (posiblemente de forma indirecta) para obtener la dirección de la ISR, la CPU sitúa en la pila el contexto actual  y inhabilita las interrupciones, y salta a la ISR.\\

7.2 Interrupciones por software: Las interrupciones por software son aquellas generadas por un programa en ejecución. Para generarlas, existen distintas instrucciones en el código máquina que permiten al programador producir una interrupción, las cuales suelen tener nemotécnicos tales como INT.
En general actúan de la siguiente manera:
Un programa en ejecución llega a una instrucción que requiere del sistema operativo para alguna tarea, por ejemplo para leer un archivo en el disco duro. En ese momento llama al sistema y se interrumpe virtualmente hasta recibir respuesta. Durante esa espera las instrucciones que se ejecutarán no serán del programa, sino del sistema operativo. Una vez éste termine su rutina ordenará reanudar la ejecución del programa interrumpido que estaba en espera y este se reanuda. \\
7.3 Excepciones: Son aquellas que se producen de forma síncrona a la ejecución del procesador y por tanto podrían predecirse si se analiza con detenimiento. Normalmente son causadas al realizarse operaciones no permitidas tales como la división entre cero o el acceso a una posición de memoria no permitida, etc. [1]

\section{APLICACIÓN}
Las interrupciones de hardware, también conocidas como INT0 e INT1, llaman a una rutina de servicio de interrupción cuando algo sucede con uno de los pines asociados.[4]  
EJEMPLO:
Se implementará una interrupción de hardware en arduino a través de un interruptor; en cuanto el pin 2 detecta una subida de voltaje, apaga el LED conectado al pin 12 y enciende el conectado al pin 13. Se usa attachInterrupt () y un delay para poder evidenciar el cambio, pues este sucede en cuestión de microsegundos y no es posible percibirlo.
\\
El circuito y código correspondiente al ejemplo podrá visualizarse al ingresar al siguiente link: \\
(https://www.tinkercad.com/things/3UKhRgfGfcR)
\section{ANÁLISIS}
Las interrupciones se utilizan para rectificar algún error o hacer algo tan simple como leer un movimiento de tecla o mouse. Cuando la CPU recibe la señal, le pide al sistema operativo que la grabe. Las interrupciones son la razón por la cual las computadoras modernas pueden realizar múltiples tareas. [5] \\
El número de interrupciones de hardware está limitado por el número de solicitud de interrupción (IRQ) líneas al procesador, pero puede haber cientos de diferentes interrupciones de software.



\section{BIBLIOGRAFÍA}

[1] Interrupción, (2020, 28 de junio), Wikipedia, La enciclopedia libre.      
[2] Microcontrolador vs Microprocesador, (2015,29 de marzo), Wordpress, aprendiendo arduino. 
[3] Interrupciones, (2009, 15 de marzo), CSL-UNEFA
[4] Interruociones (2018), Wordpress, aprendiendo arduino. 
[5] “interrupciones del sistema” en mi PC con Windows 10, (2018,07), 1000 tips informáticos.



\end{document}
